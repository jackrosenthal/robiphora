\documentclass[12pt]{article}

\usepackage{fontspec}
\setmainfont{TeX Gyre Termes}
\setsansfont{TeX Gyre Heros}
\setmonofont{TeX Gyre Cursor}

\usepackage{amsmath}
\usepackage{amssymb}
\usepackage{csquotes}

\usepackage[letterpaper]{geometry}
\usepackage{lastpage}
\usepackage{graphicx, wrapfig, subcaption, setspace, booktabs}
\usepackage[font=small, labelfont=bf]{caption}

\usepackage{algorithm}
\usepackage{algpseudocode}

\usepackage[english]{babel}
\usepackage[style=trad-plain,backend=biber]{biblatex}
\addbibresource{report.bib}

\usepackage{sectsty}
\usepackage{hyperref}
\usepackage{listings}
\lstset{language=Lisp,columns=flexible}

\usepackage{latexsym}
\usepackage{xspace}
\usepackage{epsfig}
\setlength {\topmargin} {0 mm}
\setlength {\headsep} {0 mm}
\setlength {\headheight} {0 in}
\setlength {\voffset} {0 mm}
\setlength {\oddsidemargin} {0 mm}
\setlength {\evensidemargin} {0 mm}
\setlength {\hoffset} {0 mm}
\setlength {\textwidth} {6.5 in}
\setlength {\textheight} {9 in}
\sectionfont{\scshape}

\newcommand{\HRule}[1]{\rule{\linewidth}{#1}}
\setcounter{tocdepth}{5}
\setcounter{secnumdepth}{5}

\title{Robiphora: Probabilistic Anaphora and Exophora Resolution for Robots}

\author{%
Jack Rosenthal \\
Colorado School of Mines \\
Computer Science Department \\
\texttt{jrosenth@mines.edu}}
\date{}

\begin{document}

\maketitle

\section{Introduction}

\textbf{Anaphora} is a reference from one phrase to another in a linguistical
context. \textbf{Exophora} is a reference from a phrase in a sentence to
something outside of the text, for example, objects in the real world. Both of
these linguistical concepts are important for robots that speak with humans to
understand and be able to interpret.

Often times, the referent of a given anaphor can be determined based on
knowledge about properties of the referent and the anaphor, and how they
relate. For example, consider the following sentence:
\begin{center}
    $\underbrace{\text{The house}}_{\text{referent}}$ was cold because
    $\underbrace{\text{the window}}_{\text{anaphor}}$ was open.
\end{center}
In this sentence, \emph{the window} probably refers to \emph{the house} (it is
said to be the window \emph{of the house}) since houses often have windows. In
fact, we may almost think of this in a sort of inherital pattern: \emph{the
house} is an instance of \emph{a house}, \emph{houses} are a certain form of
\emph{buildings}, \emph{buildings} often have \emph{windows}, and \emph{the
window} is an instance of \emph{a window}.

As another example, consider this sentence:
\begin{center}
    $\underbrace{\text{John}}_{\text{referent}}$ was disappointed since Sally
    did not show up to the movie with
    $\underbrace{\text{him}}_{\text{anaphor}}$.
\end{center}
In this sentence, \emph{him} could refer to a number of different noun phrases:
\emph{the movie}, \emph{Sally}, or \emph{John}. To deduce that \emph{him}
actually refers to \emph{John}, we would have to know that \emph{John} is an
instance of a \emph{man}, all \emph{men} are \emph{males}, and that
\emph{males} can be referred to by the pronoun \emph{him}. Once again, this
inherital structure came up as a common pattern in our resolution.

The goal of the Robiphora system is to provide reference resolution for not
only the linguistical context of a sentence, using anaphora, but also
considering objects contained in context of the world, which is particularly
important to using linguistics in the robotics domain.

The remainder of this paper is organized as follows: section~\ref{sec:related}
discusses related work in this area, and why Robiphora presents a novel
extension to existing works. Section~\ref{sec:details} discusses the
implementation details of Robiphora, including a formal definition of the
Object Property Definition Language, a declarative programming language
created for the Robiphora system, such so that the language could be
reimplemented upon any computer programming language or software framework.
Section~\ref{sec:conclusion} draws conclusions from this work and discusses
future work that could be done in this area.

\section{Motivation and Related Work}
\label{sec:related}

Anaphora reference resolution is a well studied problem in computational
linguistics. \cite{hobbs78} proposes a resolution system for pronouns based on
parse trees, and \cite{lappin94} adds salience measures to the syntactic
parse tree to obtain better accuracy in resolving pronoun references. Many
people have studied parsing more cases of anaphora than just pronouns, for
example, in \cite{alkofahi99}.

Beyond just anaphora resolution, exophora is a well studied field as well:
\cite{pineda00} proposes a theory for multimodal reference resoulution, and
works like \cite{prasov08} even incorporate eye-gaze to assist in resolving
objects placed in a 3-dimensional space.

Finally, \cite{pineda97} proposes a logic-based reference resolution system
which could be implemented using logic programming. Further, \cite{chai04}
proposes a graph-matching algorithm for probabilistic reference resolution.
Robiphora is an attempt to combine these two ideas: by using ideas similar to
logic programming that shows clear relationships between objects and their
types with a probabilistic reference scoring system, references can be resolved
using a technique which has not been done before.

\section{Details of the Robiphora System}
\label{sec:details}

\subsection{Description of Solution}

Robiphora is composed of a few major components:
\begin{itemize}
    \item The Object Property Description Language, the fundamental component
        of critical to Robiphora which provides mechanisms to declare
        properties about types of objects and how they relate to each other.
        Further, the language allows for declarations of what objects exist in
        the world, providing context for the listener.
    \item A probabilistic knowledge base, which provides high level interfaces
        to query the probabilistic measures that one type shares relations with
        another.
    \item A resolution component, which takes sentence semantical information
        and feeds it into the probabilistic knowledge base to discover
        references.
\end{itemize}

\subsubsection{The Object Property Description Language}

The \textbf{Object Property Description Language} (OPDL) is a domain-specific
declarative programming language created for use in the Robiphora system. OPDL
allows for the definition of \textbf{objects} in a given domain which have a
certain set of types. \textbf{Types} define an inherital structure of relation
between objects, and the nouns, pronouns, and adjectives that could be used to
refer to objects of their type. OPDL allows for multiple inheritance of types:
objects can have multiple types, and their types can have multiple base types.

OPDL uses s-expressions as it's basic form of syntax. \textbf{S-expressions}
are surrounded in parenthesis (\texttt{(} and \texttt{)}) and can contain any
number of s-expressions, symbols, or literals (the set of all of these types
refers to \textbf{expressions}), as well as any number of keyword arguments.
\textbf{Symbols} are any continuous group of characters not including
parenthesis, whitespace, colons (\texttt{:}), double quotes (\texttt{"}),
single quotes (\texttt{'}), or semicolons (\texttt{;}).  \textbf{Literals} can
be either string literals or numeric literals. String literals are any group of
characters surrounded by double quotes, not terminated by the special control
sequence \texttt{\textbackslash"}. Numeric literals start with an optional
negative sign, and are followed by one or more digits, an optional decimal
point, and zero or more digits after the decimal point. Keyword arguments
define associative properties of s-expression, and are written starting with a
colon, then followed by a symbol (which defines the name of the argument), and
then followed by an expression denoting the value of the argument. Each
s-expression may have ordered arguments, as well as keyword arguments.
\textbf{Comments} start with a semicolon not inside of a string literal.
Comments and whitespace are ignored and have no semantical meaning to the
language.

At the outermost level of expression nesting, s-expressions are used to
provide definitions for objects and types, and import data from other OPDL
files. The ordering of these expressions provides no meaning to an OPDL
implementation: the meaning of the language is the same regardless of the
order. The collection of these expressions is referred to as an \textbf{OPDL
probabilistic knowledge base}.

Object definitions begin with the symbol \texttt{object} followed by the object
name, and have an optional keyword argument of \texttt{type}, which should have
an s-expression listing the types the object may be. An example object
definition is shown in Figure~\ref{opdl:object}.

\begin{figure}[ht]
    \begin{lstlisting}
    (object thehouse
      :type (house blue))
    \end{lstlisting}
    \caption{A simple object definition. This object is named \texttt{thehouse}
    and is of types \texttt{house} and \texttt{blue}.}
    \label{opdl:object}
\end{figure}

Type definitions begin with the symbol \texttt{type} followed by the type name,
and have the following optional keyword arguments:
\begin{enumerate}
    \item \texttt{bases}, an s-expression which defines the set of base types
        that the type will inherit from.
    \item \texttt{antibases}, an s-expression which defines which base types
        should cause low probability of showing inheritance.
    \item \texttt{collection-of}, an s-expression that defines that the type
        refers to multiple of a certain type (for example, the \texttt{people}
        type might be a collection of \texttt{human}).
    \item \texttt{nouns}, an s-expression of string literals that contains all
        of the nouns which can be used to refer to objects of that type. This
        should only be used to specify strict nouns: pronouns and other
        components of noun phrases (such as adjectives) are specified
        separately.
    \item \texttt{pronouns}, an s-expression of string literals that contains
        the pronouns that can be used to reference objects of this type (for
        example, the \texttt{male} type might have pronouns ``he'', ``him'',
        and ``himself'').
    \item \texttt{provides-adjectives}, an s-expression of string literals that
        specifies what adjectives might increase the probability of reference
        to this object.
\end{enumerate}

An relatively simple example set of related type declarations is shown in
Figure~\ref{opdl:types}.

\begin{figure}[htb]
    \begin{lstlisting}
    (type human
      :nouns ("human"))

    (type speaker
      :bases (human)
      :pronouns ("i" "me"))

    (type listener
      :pronouns ("you"))

    (type male
      :bases (human)
      :antibases (female)
      :nouns ("male")
      :pronouns ("he" "him" "himself"))

    (type man
      :bases (male)
      :nouns ("man"))

    (type boy
      :bases (male young)
      :nouns ("boy"))

    (type female
      :bases (human)
      :antibases (male)
      :nouns ("female")
      :pronouns ("she" "her" "herself"))
    ;; a similar set of types (girl, woman, etc) could be defined as well

    (type people
      :collection-of (human)
      :pronouns ("they" "them" "themselves"))
    \end{lstlisting}
    \caption{Types as defined in OPDL. There are quite a few types here: this
    is to demonstrate the various features of OPDL in a realistic manner.}
    \label{opdl:types}
\end{figure}

The only other OPDL declarative is \texttt{import}. The \texttt{import}
expression loads more definitions from another file, relative to the current
file's path. This allows a broad set of types to be created to define how the
world works, and many narrower sets of types and objects to be created for
specific scenarios and contexts.

The reader may be wondering at this point: \emph{why create OPDL when the same
logic can be encoded using a logic programming language like Prolog?} While the
same \emph{logic} can be encoded using Prolog, OPDL defines probabilistic
relations between objects. By using OPDL, references can be evaluated in a
well-defined and non-stochastic manner (similar to Prolog), but solve the
references using probabilities. Details of the probabilistic reference
resolution are discussed in the next section.

\subsubsection{Probability Evaluation}

In order to derive probabilities from an OPDL probabilistic knowledge base,
inherital probabilities must be assigned to the act of inheritance. Further,
probabilities must be assigned to the act of antibasing (strictly not
inheriting from) and non-basing (when it not stated whether inheritance exists
or not). The symbols $P(I)$,  $P(A)$, and $P(\lnot I)$ are used to refer to
these constants, respectively. For the Robiphora implementation, the following
constants were selected:
\begin{align*}
    P(I) &= 0.9 \\
    P(A) &= 0.0 \\
    P(\lnot I) &= 0.1
\end{align*}

Depending on the structure of the OPDL definitions, a different set of
constants may want to be selected. For example, when a set of types is selected
that uses many levels of inheritance to convey a simple meaning, it may be
desired to raise $P(I)$ or lower $P(\lnot I)$ so that the inheritance still
conveys its meaning. For $P(I) = 1.0$ and $P(A) = P(\lnot I) = 0.0$, OPDL
becomes a logical programming language.

Once we have these probabilities established, we may design an algorithm to
determine the probability that one type could have the properties of another.
It should be unsurprising to the reader that the algorithm is similar to the
DPLL Boolean variable satisfaction algorithm\cite{dpll}. The algorithm used in
Robiphora recursively determines the maximum probability of one type being
another, multiplied by the probability defined in the $P(I)$, $P(A)$, or
$P(\lnot I)$ constants. The base case is that the two types are equivalent, in
which case, the probability is $1.0$. This algorithm is shown in
Algorithm~\ref{alg:queryis}.

\begin{algorithm}
    \caption{Query the probability that an object of type $a$ could be referred
    to by the properties of type $b$ within an OPDL knowledgebase}
    \label{alg:queryis}

    \begin{algorithmic}
        \Procedure{TypeProbability}{$K, a, b, v$}
            \Comment{$K$ is the knowledge base, $v$ is the set of visited nodes}
            \If{$a$ is $b$}
                \State \Return 1.0
            \EndIf
            \If{$K$ states $a$ is in $b$'s antibases}
                \State \Return $P(A)$
            \EndIf
            \State{Add $A$ to $v$}
            \State \Return{
                $\begin{aligned}
                    \max(&\{P(I) \times \hbox{\Call{TypeProbability}{$K, c, b, v$}}\ |\ c \in K_a - v\} \\
                    \cup\  &\{P(\lnot I) \times \hbox{\Call{TypeProbability}{$K, c, b, v$}}\ |\ c \in K_a^c - v\})
                \end{aligned}$}
        \EndProcedure
    \end{algorithmic}
\end{algorithm}

\subsubsection{Pruning}

Algorithm~\ref{alg:queryis} can receive significant performance improvements by
passing an extra parameter $\alpha$ that indicates the best probability we know
we can achieve so far. This is shown in Algorithm~\ref{alg:pruned}

\begin{algorithm}
    \caption{Query the probability that an object of type $a$ could be referred
    to by the properties of type $b$ within an OPDL knowledgebase, taking into
    account pruning parameters}
    \label{alg:pruned}

    \begin{algorithmic}
        \Procedure{TypeProbabilityPruned}{$K, a, b, v, \alpha$}
            \If{$a$ is $b$}
                \State \Return 1.0
            \EndIf
            \If{$K$ states $a$ is in $b$'s antibases}
                \State \Return $P(A)$
            \EndIf
            \State{Add $A$ to $v$}
            \ForAll{$c$ in $K_a$}
                \State{$p \gets P(I) \times \hbox{\Call{TypeProbabilityPruned}{$K, c, b, v, \alpha$}}$}
                \If{$p > \alpha$}
                    \State{$\alpha \gets p$}
                \EndIf
            \EndFor
            \If{$\alpha < P(\lnot I)$}
                \ForAll{$c$ in $K_a^c$}
                    \State{$p \gets P(\lnot I) \times \hbox{\Call{TypeProbabilityPruned}{$K, c, b, v, \alpha$}}$}
                    \If{$p > \alpha$}
                        \State{$\alpha \gets p$}
                    \EndIf
                \EndFor
            \EndIf
            \State\Return{$\alpha$}
        \EndProcedure
    \end{algorithmic}
\end{algorithm}

Algorithm~\ref{alg:pruned} is guaranteed to always produce the same results as
Algorithm~\ref{alg:queryis} so as long as $\alpha$ is initially selected to be
less than $\min(P(A), P(\lnot I))$ and $P(I) > P(\lnot I)$\footnote{This is a
perfectly sane assumption: systems defined to have a stronger probability of
non-inherited edges over inherited edges simply do not make sense.}, but does
so in greater or equivalent efficiently. Similar to DPLL \cite{dpllheuristic},
if the selection of the ordering of iteration through $K_a$ and $K_a^c$ is
selected using a good heuristic, the author would conjecture that performance
would be improved further\footnote{This is, however, left to others wishing to
do future work in this research area.}.

\subsubsection{Resolution}

Once the type-querying algorithm (Algorithm~\ref{alg:queryis} and
Algorithm~\ref{alg:pruned}) is defined, resolving anaphors in sentences
becomes relatively trivial: for each potential referent (or a nearby subset of
referents) to each anaphor, determine the probability that the type(s) of the
anaphor and its modifiers (such as adjectives) are referring to the referent's
type. Note that the set of referents could be determined from the situated
context of the listener or robot using OPDL, or from the sentence context (or
both).

% TODO: Pseudocode?

Robiphora makes use of Sattizahn's probabilistic combinatory categorical
grammar parser\cite{sattizahn} to create semantical parses of the sentences,
and then considers the set of all possible referents when preforming the
resolution.

\subsection{Evaluation}

Robiphora uses a non-stochastic algorithm for reference resolution, and hence,
the results will only be as good as the OPDL input is. While automatic
generation of an OPDL-like structure from a corpus text to be used with this
algorithm would be an intriguing research topic, it is most certainly not the
focus of Robiphora as of now. Further, the author is not aware of any other
reference resolution systems which use an OPDL-like structure. Hence,
evaluation of accuracy resolving real natural language compared to established
techniques needs to be done at a meta-level. See section~\ref{sec:related} for
an this overview of existing techniques and how they compare to the Robiphora
system.

However, evaluation of the time-scale complexity of Robiphora is an important
consideration. The remainder of this section will focus on deriving performance
measures for the algorithms presented in this paper.

Algorithm~\ref{alg:queryis} has a time complexity of $\Theta(2^n)$, where $n$
is the number of types in the system, when $a \ne b$ and $b$ is not an antibase
of $a$ (under these cases, it is trivial to show the time complexity is
constant). This is because, for each type, we must consider whether that type
is included or not as a measure in our probability.

When pruning is implemented, like in Algorithm~\ref{alg:pruned}, the worst case
time complexity has not changed (it is still $O(2^n)$), but the best case time
complexity becomes $o(n)$ as we may not have to consider both states of each
type in our analysis. Given a good ordering heuristic (as stipulated earlier in
the paper), the author speculates this could happen on a wide variety of
practical inputs. Compared to the truth-table like algorithm presented in
\cite{chai04}, which has a lower bound of $o(2^n)$, Algorithm~\ref{alg:pruned}
now has a much more time-efficient lower bound.

\subsection{Discussion}

While Robiphora has OPDL as it's biggest advantage: a mechanism to specify how
the world is typed in a well-defined and non-stochastic manner, this is also
Robiphora's biggest disadvantage. Modelling the behavior of the world is a hard
problem, and when you have to do so in a well-defined manner, it's remarkably
harder.

Perhaps some of this could be evaded by developing a technique to automatically
generate OPDL-like semantics from a corpus text or capture of a real-world
scenario. Such a problem is most certainly not trivial though, so the author
leaves this to be tackled by a future researcher.

Other possibilities for future work in this area include designing ordering
heuristics for Algorithm~\ref{alg:pruned}, or extending OPDL to allow
specification of explicit probability edges.

\section{Conclusion}

Robiphora introduces a new and novel technique for multi-modal reference
resolution based on a probabilistic model. Along with this, Robiphora
introduces a few new concepts to the broader computer science domain as well:
the Object Property Definition Language, a domain-specific declarative
programming language, the probabilistic object-type resolution algorithm (see
Algorithm~\ref{alg:queryis}) and the associated pruning techniques (see
Algorithm~\ref{alg:pruned}). The author hopes that future researchers are able
to integrate the techniques presented in this paper in their own work.

\label{sec:conclusion}

\subsection{Acknowledgments}

The author would like to make a special acknowledgment to Paul Sattizahn for
his work on his semantical parser using a probabilistic combinatory categorical
grammar. He was willing to integrate his parser into the Robiphora framework
before it was even working, and even willing to integrate an early version of
OPDL into his probability evaluation function.

In addition, the author would like to acknowledge Tom Williams for his
excellent teaching which led to the development of Robiphora. Without his
Linguistic Human Robot Interaction course, Robiphora would not even be a
concept in the author's mind.

%%%%%%%%%%%%%%%%%%
\clearpage
\printbibliography
%%%%%%%%%%%%%%%%%%

\end{document}
